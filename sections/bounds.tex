\section{Situace definující hranice systému}

	\subsection{Ideální scénář}

	V ideálním scénáři vše probíhá hladce a~uživatelé zažívají bezproblémový průběh od objednání jídla až po jeho doručení.

    	\subsubsection{Objednávka} 

		Zákazník otevře mobilní aplikaci a~jednoduše vybere své oblíbené jídlo z~nabídky restaurací v~okolí. Díky intuitivnímu rozhraní a~rychlému načítání má přehled o~dostupných možnostech a~snadno provede objednávku ve svém oblíbeném podniku.

    	\subsubsection{Potvrzení objednávky} 

		Poté, co zákazník dokončí objednávku, systém okamžitě potvrdí jeho transakci a~zajišťuje, že informace o~objednávce jsou předány restauraci a~doručovateli v~reálném čase. Obdrží potvrzovací e-mail s~detaily objednávky a~očekávaným časem doručení.

    	\subsubsection{Přiřazení doručovatele} 

		Doručovatel obdrží notifikaci o~nové objednávce a~okamžitě vyrazí na cestu k~zákazníkovi. Systém automaticky vybírá nejvhodnějšího doručovatele na základě jeho dostupnosti a~vzdálenosti od místa doručení, což minimalizuje čekací dobu.

    	\subsubsection{Dodání objednávky} 

		Doručovatel dorazí k~zákazníkovi včas, předá mu jeho objednávku a~případně vyřeší jakékoliv dotazy či požadavky. Zákazník je spokojený s~rychlým a~profesionálním přístupem doručovatele a~s kvalitou jídla. Jeho zážitek je pozitivní a~motivuje ho k~opakovanému využití služby.
	
	\subsection{Hraničně řešitelný scénář}

	V této situaci se vyskytují menší technické problémy, které mohou mít vliv na plynulost procesu objednávání a~doručování jídla.

    	\subsubsection{Zpoždění v~potvrzení objednávky} 

		Během večerní špičky, kdy je vysoký objem objednávek, může dojít k~dočasnému zpomalení systému a~zpoždění v~potvrzení objednávek. Technický tým monitoruje výkon systému a~provádí nezbytné úpravy, aby minimalizoval dopady na uživatele a~zajistil co nejlepší uživatelský zážitek.

    	\subsubsection{Nedostatek dostupných doručovatelů} 

		V některých okamžicích může být omezený počet dostupných doručovatelů, což může vést k~prodlouženým dodacím lhůtám. Systém se snaží zajistit, aby doručování probíhalo co nejrychleji, a~případně informuje uživatele o~odhadovaném čase doručení.

    	\subsubsection{Chybné údaje uživatele} 

		Někdy se může stát, že uživatel zadá chybné údaje nebo nekompletní informace, což může způsobit potíže s~doručením objednávky. Systém provádí kontrolu zadaných údajů a~pokud dojde k~detekci neplatných informací, upozorní uživatele a~umožní jim jejich opravu před odesláním objednávky.

	\subsection{Situace, které systém nezvládne vyřešit}

	Přestože systém je navržen tak, aby byl robustní a~spolehlivý, existují mimořádné události, které mohou překročit jeho možnosti a~vyžadovat manuální zásah nebo alternativní řešení.

    	\subsubsection{Výpadek elektrického proudu} 

		V případě výpadku elektrického proudu může být provoz systému přerušen, což by mohlo mít za následek nedostupnost služby a~ztrátu dat. Systém je vybaven záložními zdroji energie a~plánem obnovy dat, ale v~extrémních případech může dojít k~dočasnému vypnutí služby až do obnovení napájení.

    	\subsubsection{Kybernetický útok} 

		Útoky hackerů mohou ohrozit bezpečnost systému a~vést k~poškození dat nebo ztrátě citlivých informací. Systém je vybaven bezpečnostními mechanismy pro ochranu proti různým formám útoků, ale v~případě úspěšného útoku může dojít k~dočasnému vypnutí služby a~vyžadovat rozsáhlé obnovovací opatření.

    	\subsubsection{Přírodní katastrofy} 

		Nepředvídatelné přírodní katastrofy, jako jsou zemětřesení, povodně nebo bouře, mohou poškodit infrastrukturu systému a~způsobit výpadek služeb. Systém musí mít plán kontinuity provozu a~schopnost rychle se adaptovat na změněné podmínky, aby minimalizoval dopady na uživatele a~zajišťoval co nejrychlejší obnovení služby po krizové situaci.
	
\pagebreak
